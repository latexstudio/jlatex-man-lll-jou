%#!platex jou.tex
\chapter{まえがき}\chaplab{preface}

%\begin{abstract}
%\TeX というプログラムに魔物が棲んでいるという噂,どうやら本当みたいです.
%\end{abstract}

\newcommand*\kugiri{\bigskip}

\section*{これは何のための本か}

\zindind{文書}{の執筆}
\indindz{レポート}{科学技術系の}
\indindz{論文}{科学技術系の}
何らかの\Z{文書}を執筆するときに,まず\kenten{何を}書くべきかという内容に関して
考えると思います.しかし,書くべき内容が決まったとしても\kenten{どのように}
書くかは必ずしも決まりきったものではないと思います.大変大きな文書であった
り,数式を大量に含むようなものであれば,何かしらの包括的な方法があれば便利
でしょう.特に科学技術系のレポートや論文を執筆する事を考えると,それに特
%化したものを使った方が利便性も向上するでしょう.このような状況で広く使わ
化した手法を用いた方が利便性も向上するでしょう.このような状況で広く使わ
れているのが \ruby{\LaTeX}{ラテツク} と呼ばれるプログラムです.

\indindz{文書}{体裁の整った}
この本では \LaTeX を用いた文書作成について解説します.\LaTeX は
\ruby{\TeX}{テツク} と呼ばれるプログラムの上に構築されているシステムで
す.\laTEX はフリーウェアであり,誰でも無料で自由 (\emph{free}) に入手
する事が可能です.\LaTeX をうまく使いこなせば,\Z{体裁}の整った美し
い文書が簡単に作成できるようになります.

\zindind{文書}{の論理構造}
\zindind{書式}{と内容の分離}
科学技術系の文書を執筆している時,本来ならば「見出しはゴシック体で 24\,pt」
であるとか「1行の文字数は 40 文字で 1 ページは 36 行」という様な\Z{書式}に
関する問題は考えない方が良い場合もあり,\LaTeX ではこれが可能です.
\LaTeX では文書の論理構造に気をつけながら原稿を執筆する事ができるため,
書式と内容を分離する事が可能なのです\footnote{\Z{HTML} \& \Z{CSS}や\Z{XML}
\& \Z{XSL}のような関係と似ていると思って良いでしょう.}.

\zindind{体裁}{の調整}
本書は全く \LaTeX を使った事がない人を対象に,すでに必要とされる書式が
整っている段階\footnote{比較的規模の大きな学会等であれば,論文投稿におけ
る \LaTeX の書式を提供している所が数多くあります.大学等の教育機関でも
\LaTeX 用の学位論文のスタイルを提供している所もあります.}で,レポートや
論文等の文書を \LaTeX を用いてどのように執筆するかを解説するのが目的とな
ります\footnote{体裁を調整するというのは,本来ならば執筆者が担当する部分
ではなく,投稿を求める方が行うべき作業ですから,体裁調整に関する説明は応
急処置的な領域に留めています.}.

この本は単に「\LaTeX というプログラムの機能を紹介した\kenten{説明書}」と
いうよりは%\footnote{「できる○○」シリーズというのは具体的な手順・操作
%の方法だけしか触れておらず,その原理的な構造に関する理解への言及が非常に
%少ないというのが一般的な説明書の姿だと思います.そのような説明書が必要ない
%とは思っていません.あくまで役割に関する問題だと捉えています.}
,「動作原理や仕組み理解して \LaTeX を使いこなすための
\kenten{教科書}」に近いと思います%\footnote{そのため「\Z{情報リテラシー}
%入門」のような大学講義で十分活用できる内容に近いと思っています.情報を伝
%えるためのプリミティブな手段は,情報技術がいくら発達しても人間の脳は文字
%を基本とした媒体による伝播が一番低コストで確実な方法だと,私は思っていま
%す.この本は文字を基本とした情報交換(文章表現能力)の基礎となる部分も汲み取る
%ように努力しています.}
.昨今は \Z{tips}集のような説明書が重宝されがちです
が,このような説明書では自力で問題を解決する能力の獲得や,更なる飛躍へ向け
たステップアップが難しいという側面があります.
%
%\TeX や \LaTeX は何も数式の再現性が素晴らしいからとか,文書の論理構造が
%明確であるという状況だけに使われるシステムではないと思います.
\TeX や \LaTeX は何も数式の再現性が素晴らしいからとか,
文書に明確な論理構造が求められる状況に効率よく対応するためだけに
使われるシステムではないと思います.
\zindind{ファイル}{入出力}%
\zindind{コンパイル}{の回数}%
%
%文章を紙面に構成する時の\Z{物理的制約},\Z{データ型}と\Z{四則演算},
%\Z{アロケーション},ファイル入出力,プログラムの高速処理のための並列性と
%最適化,ラベルの後方参照とコンパイル回数について理解する事ができます
%
本書よりも高度な内容に関しては,著者がウェブページで公開してい
る『好き好き\LaTeXe』シリーズを参照してください\footnote{%
\webThorTypo}.\genzai 多くの続編は未完ですが,草稿段階の文書を暫
定的に公開しています.



\section*{凡例}

\index{凡例}
本書では\Z{書体}を変更する事によって同じ語句でも
違った意味を持つものが多数あります.\qu{\prog{dvipdfmx}}
という語があったとしても\qu{\textsf{dvipdfmx}}や\qu{\texttt{dvipdfmx}},
\qu{\textsl{dvipdfmx}},\qu{\textit{dvipdfmx}}はすべて別の意味を
持っています.\zindind{書体}{の種類}これらの書体の種類については
\secref{font}を参照してください.
\begin{center}
\zindind{キーボード}{からの入力}%
 \begin{tabular}{lll}
 \TR
 \Th{書体}          & \Th{意味}      & \Th{例} \\
 \MR
 ローマン体    & 通常の文章& \textrm{dvipdfmx}\\
 サンセリフ体  & パッケージやクラス\pp{\secref{class}参照}& \textsf{dvipdfmx}\\
 タイプライタ体& キーボードからの入力など& \texttt{dvipdfmx}\\
 イタリック体  & 変数や強調& \textit{dvipdfmx}\\
 スラント体    & オプション\pp{\secref{classopt}参照}& \textsl{dvipdfmx}\\
 \BR
 \end{tabular}
\end{center}

本文中で左側にタイプライタ体,右側にそれに準じた出力例が
あるものは,入出力の対を表します.
\par\addvspace{3.0ex plus 0.8ex minus 0.5ex}\vskip-\parskip%
\hspace*{\IOm}\hspace*{-1ex}%
\makebox[0pt][l]{%
  {\begin{minipage}[c]{.47\fullwidth}\small%
\begin{ttfamily}%
The length of a pen should be\\
comrotable to write with: too\\
long and it makes him tired;\\
too short and it\cmd{ldots}.%
\end{ttfamily}%
  \end{minipage}%
  }\hspace{0.05\fullwidth}%
  {\begin{minipage}{.47\fullwidth}%
      \begin{trivlist}\item\small%
The length of a pen should be comrotable
to write with: too long and it makes him
tired; too short and it\ldots.%
   \end{trivlist}%
   \end{minipage}}}%
\par\addvspace{3.0ex plus 0.8ex minus 0.5ex}\vskip-\parskip%

\Z{テキストエディッタ}などを使い,原稿ファイルで左側のように入力すると,
右側の出力例と同じような結果を確認できます.

本文中の入出力例に対しては,ただ眺めるのではなく,実際に自分で入力し,
実行結果を吟味してみる事をお勧めします.

なぜ実行結果を吟味する事を勧めるのか,それはコンピュータプログラム
というのは他の分野に比べると現象の再現性が確定的であり,追試可能性が
高いという点にあります\footnote{心理学実験のように被験者を集める必要性も
ありません,恐らくある程度の処理能力を有する計算機が一台あれば十分です.}.

科学的な論述が中心の場合は「論拠がなければ信用できない」というのが筋です
が,コンピュータプログラムの場合は,実際に手を使ってプログラムを組んで
みて,コンパイルし,実行(実験)し,結果を吟味(考察)する事が誰でも
自由にできます(そのプログラムのソースコードが呈示されている限り).
さらに数学の証明のように,アルゴリズム的な話は一度自分でその構造と原理を
考えてプログラムすれば,%適切な解釈ができるようになり,そこから五感を使って
得られた知識は一生その人のものになるのです.このような考えに基づき,本
書では例を中心に話を進めているので,その例を自分で入力し,さらにはその実
行結果を吟味する事を推奨します.

\kugiri

文中において\type{which perl}という表記は\Z{コマンドプロンプト}や
\Z{シェル}などの\Z{コンソール}からの入力を示します.
複数行の入力の場合は次のようにしています.

\begin{InTerm}
 \type{platex input.tex} 
 \type{jbibtex input.tex} 
 \type{dvipdfmx -S -o output.pdf input.dvi}
\end{InTerm}

\index{"$@\verb+$+!コンソールの\zdash}%"
先頭のドル`\str$'はコンソールに表示されている記号で,
ユーザは入力しません.

\kugiri

\Z{キーボード}上の特定の\Z{キートップ}を押す事を示すには \key{Alt} の
ようにしています.\key{Ctrl,Alt,Delete} は \key{Ctrl},\key{Alt},
\key{Delete} キーを同時に押す事になります.\key{Ctrl,x}\key{Ctrl,s} は 
\key{Ctrl,x} を押した後に \key{Ctrl,s} を押す事を表します.

\kugiri

本書に表記されている\Z{バックスラッシュ}`\bs'はWindows環境によっては,
テキストエディッタ等で円記号`\yen'として視認できると思います.Windows
ユーザの方は基本的に`\yen'が\unixos では`\bs'に文字化けしていると認識
して頂いて問題ありません.

\kugiri

\index{変数}
何らかの文字列や数値に置き換わるものは\va{変数}のように
表記しています.

\kugiri

\begin{Trick}
\begin{normalsize}
  少々難解だと思われる箇所,\LaTeX の動作原理に触れている段落に
 関しては,この段落のように『急カーブあり危険!』の\textdbend
 マークを付与しています.
\end{normalsize}
\end{Trick}


%\begin{comment}

\section*{フリーウェアとは}

{\LaTeX}はフリーウェアですがその重要なマニュアルはフリーではありません.
{\LaTeX}プロジェクトメンバーの\Person{Michel}{Goossens}や
\Person{Sebastian}{Rahtz},\Person{Frank}{Mittelbach},
\Person{Leslie}{Lamport}らが出版しているマニュアルは日本語訳で1冊5,000円
程度の値段です.そこで{\LaTeX}ユーザが必携といわれている書籍は4冊程あ
ります.

\begin{itemize}
\item \wasyo{\LMANUAL}\cite{latexbook}\quad 3,000円.
\item \wasyo{\COMP}\cite{latexcomp}\quad 4,800円.
\item \wasyo{\GCOMP}\cite{graphicscomp}\quad 5,400円.
\item \wasyo{\WCOMP}\cite{webcomp}\quad 4,800円.
\end{itemize}

\yo{必携の本を買ったら18,000円もかかるのか}と思われる事でしょう.
これではフリーウェアだから使ってみようと思った方や,誰かに薦められて使い
始めた方は手を出しづらいのではないかと思います.また,{\LaTeX}の使い方で
ある技術資料は全て公開されていますので,親切な方がウェブページなどで詳し
く取り扱っている場合があります.そのようなページを見れば特に困る事はな
いと思いますが,情報が離れ離れで存在するので,どうも勝手が悪いようです.

\Person{Richard}{Stallman}の訴える通り,これがフリーウェアの抱える問題点
ではないかと思います.そこで新たにフリーなマニュアルを作成する事にしま
した.ただし{\LaTeX}の既存のマクロ,クラスならびにプログラムの活用方法に
ついての話に限定します.事務的な書類の作成ではなく主にレポートや論文を書
くための情報を集めていますので,表に色を付けたいとか,フォントにこだわり
たいという情報は含んでいません.さらにマクロ・クラスの作成方法は最小限に
とどめていますので,既存の良書を\appref{info}から参照してください.

%\end{comment} 

