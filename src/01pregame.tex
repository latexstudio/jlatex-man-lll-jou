%#!platex jou.tex
\chapter{執筆を始める前に}\chaplab{pregame}
\begin{abstract}
\LaTeX というプログラムと文章を紙面に構成する上で重要となる組版というも
のについて少し紹介します.また,\LaTeX の歴史的背景と諸事情についても簡
単に触れておきます.
\end{abstract}

\section{組版とはなんだろうか}
\index{組版}\ruby{組版}{くみはん}とはある媒体,特に書籍などの紙のうえに
読者が読みやすいように必要な情報を適切な位置に配置する事です.

現代ではコンピュータ上で文書を組版できるようになりました.だれでも手軽に
印刷用の美しいフォントを用いた組版が可能です.ここで文書がどのようにして
組版されているのかを少し説明します.

世界中で出版されている書籍は一定のルールに沿って組版されているものです.
例えば1行を何文字にするか,1ページを何行にするかなどの約束事があります.
このような様式をどのようにするのかは各出版社や各種学会の組織が各々で定め
ています.

なぜこのような決まり事があるかというと,文字や図を含む本や雑誌は必ず誰か
に見てもらう,読者を相手にしている事を前提としているからです.その本の
内容に合わせて読者にとって読みやすい本とは何かを追求してこのような様々な
書式が存在します.

% column
%{\LaTeX}を用いるとユーザがそのような高度な技術を持っていなくてもプログ
%ラムが半自動的に組版するようになっています.しかし最低限のルールを覚えな
%ければとても出たら目な文書に仕上がってしまいます.
%
%世の中にはパソコンで動くワープロソフトと呼ばれるソフトウェアが多数存在す
%るようです.OpenOffice.orgのWriterとかMicrosoft Office Wordなどがその類
%です.これらのソフトと組版ソフトである{\LaTeX}は何が違うのでしょう.それ
%をこれからじっくり眺めていくことにしましょう.まずは{\LaTeX}の周辺と予備
%知識を説明します.

\section{文章表現}

\latexno{の最低限の規則}%
\indindz{文書}{でたらめな}%
{\LaTeX}を用いるとユーザがそのような高度な技術を持っていなくてもプログ
ラムが半自動的に組版するようになっています.しかし最低限のルールを覚えな
ければ,\K{とても出たら目な文書に仕上がってしまいます}.

次の例文の中には多くの\Z{文章表記}の上での約束事が秘められています.
\begin{InOut}
The length of a pen should be 
comrotable to write with: too 
long and it makes him tired; 
too short and it\ldots.\par
When I was a young---a foolish 
boy---the pen was too long! So 
I used to break it.
\end{InOut}
ここでは句読点とダッシュの用法が確認できます.
コロン,セミコロンなどの記号はコンマ,ピリオドと同様に,記号の前に空白
(空き)を入れず,後ろに\zindind{記号}{の前の空白}半角の空白を挿入してい
ます.文を中断するダッシュ,\ruby{em-dash}{エムダッシュ} の場合は前後に
空白を入れません.
\begin{InOut}
``\,`Stop!' the man said.'' \par
Prof.~Albert Einstein (1897--1955) 
was born in German (see fig.~3).
His famous equation $ E = mc^2 $ 
is written in the theory.
\end{InOut}
クオートで一文を引用していますが,引用の中の引用とクオートが隣接している
部分は若干の空白を挿入しています.アインシュタインが1897年から1955年まで
生きていたという,数値の範囲を示す場合は \ruby{en-dash}{エンダッシュ} を
用います.日本語でも波ダーシ`〜'は使いません.「図~3を参照せよ」という意
味の `(see fig.~3)'ですが,丸括弧(パーレン)の左側(起こし)に空白を入
れていますが,右側(受け)には入れていません.`fig.'と`3'のあいだで改行する
事は好ましくないので,チルダ`\str~'を補っています.数式中の等号`\str='
は関係演算子を意味していますので,前後に適切な空白が挿入される事になり
ます.
\begin{InOut}
$$ agenda \leftarrow office $$
$$ \mathit{agenda} \leftarrow 
   \mathit{office}$$
\end{InOut}
上記の二つの例はいずれもアルゴリズムです.しかし,二つ目は
正しい意味なのですが,一つ目は間違った意味になっています.
執筆者の意図としては「リスト$\mathit{agenda}$に$\mathit{office}$を
代入する」という事になりますが,一つ目は \C{mathit} という
コマンドを使っていないために,「変数$a$, $g$, $e$, $n$, $d$, $a$の
積に変数$o$, $f$, $f$, $i$, $c$, $e$の積を代入する」という全く
異なった意味になってしまいます.

\zindind{文章}{表現}%
このように文章表現を行う上では\Z{作文}(と\Z{組版})に関する約束事・
知識を知らなければ\K{読者に正確な意図が伝わらなくなります}.

\zindind{記号}{の意味}%
\zindind{記号}{の使い方}%
%TODO なんかちょっと文章が変だよね.
他とのコミュニケーションにおいて\KY{文字}による伝達を採用する
場合,それらに用いる記号の意味を正確に把握しなければ,「間違っ
た意味」が相手に伝わる事になります.文書の正確性が保持されて
いなければ,読者の深い理解と共感を得る事が難しくなります.
\zindind{文書}{の正確性}%

\zindind{記号}{の使い方}
本書でもそのような「記号の使い方」に関する部分を取り扱い,それらを\LaTeX
上でどのように実現すれば良いのかも説明します.このような文章表現に関する
部分は\LaTeX\ を用いない場合においても重要であると考えますので,本文中で
\KY{強調}して表記しています.

\indindz{文書}{ワープロソフトによる}%
近年は\Z{ワープロソフト}と呼ばれるソフトウェアが多数存在します.
\Prog{OpenOffice.org}の\Prog{Writer}や\Prog{Microsoft Office}の
\Prog{Word}などがその類です.これらのワープロソフトと\LaTeX のあいだには決定
的な差があります.ワープロソフトは文書の要素に直接視覚的な調整を施します.
例えば,`I'という文字をワープロソフトで\Z{斜体} (\textit{I}) にすると,
強調を意味するのか変数を意味するのかという部分が曖昧になります.\LaTeX
をうまく使いこなせば,視覚的にその文字の意味を認識しながら,文書を執筆す
る事ができるようになります.


\section{{\TeX}とは何か}\seclab{tex}

\index{TeX@\TeX}\index{プログラム!TeX@\TeX}
\ruby{\TeX}{テツク}~\cite{texbook}とは\Person{Donald}{Knuth}によって開発さ
れた組版プログラムです.特筆すべき点は数式の処理に優れている事,簡単
なレポートの作成から論文の作成,果ては\Z{商業出版}にも耐えう
る機能を持っている事などです.

% column: いわゆるどうでも良い話
%\begin{metacomment}
% 良く \TeX が開発された謂れをここで記述する事が多いようだが,そんなこと
% は単なる \LaTeX 使いにはどうでも良い話であろう.あえて私はそういうコラ
% ム的な話題は掲載しない事とした.しかし,物好きのために原稿のコメントだ
% けにはそれを残しておく.
%\end{metacomment}
%
% \usepackage{mflogo}
%
% どうして Knuth が \TeX を開発したのか,直接的な原因は金銭的な余裕がな
% くなったためです.Knuth が大著 TAOCP: \emph{The Art of Computer
% Programming} を執筆している段階で,Knuth がお気に入りの Monotype で
% 良質な組版をしてくれる植字工(活字職人)がいる印刷所が潰れてしまった.
% それ以降のシリーズの刊行はコンピュータによる組版で済ませた.しかし,
% その品質は熟練した植字工が仕上げた美しい書籍に比べれば,愕然とするもの
% だった.自著のタイトルが \emph{The Art of Computer Programming}である
% 以上,その組版も高品質でなければ世に出すのは忍びなかったであろう.
% しかし,当時はコンピュータによる安価な電子出版技術の波に押されて,
% 活版印刷では出版の採算が取れなくなってきていたのである.Knuth 以外の
% 大学教員は典型的に IBM の計算機で数式組版を行っていた.この数式組版の
% 品質は非常に劣ると感じたため,自分で組版システムを作成する決心をした.
% これが 1977 年の事である.しかし,Knuth は組版システム \TeX の開発をし
% ているときに,気づいた事がある.それは「良質な組版には良質な書体が必要
% である」というい事実である.この事実は Knuth が Monotype による活版印
% 刷を体験していたために,顕著であったと思われる.そこで,コンピュータに
% よる今までにないメタな書体を作成する事ができる {\MF} を開発する事とな
% る.1982 年には現在のグリフ(字形)に近い Computer Modern フォントが作
% 成された.このとき,実は Knuth に Helmann Zapf が指導していたという話
% も残っている.実際に Zaph が Stanford 大学まで何度も足を運ぶという事が
% あったらしい(逆の事実は把握していない).付加価値としてこのとき,
% Knuth は大規模なコードを書くときにはそれを管理するための何らかの清書用
% のラッパーが必要であるとも感じた.プログラムのソースコードにコメントを
% 書くのは日常的に行われている事だが,実際,他の人にコードを見せるときに
% は,もっと教科書的に学習ができるような体裁になっていた方が便利だろうし,
% 再利用性も向上すると考えた.これが WEB と呼ばれる物である.WEB では
% 「コメント+ソースコード」を書くときのルールがあり,その文法でソースを
% 記述する事により,コードの可読性を高めていた.実際には tangle というプ
% ログラムが WEB コードからコンパイルに必要なソースコード部分を抽出し,
% weave が「コメント+ソースコード」からマニュアルとなる文書を生成する.
% もちろん,このマニュアルは \TeX でタイプセットできるようになっている.
%
% \TeX というのは 20 世紀の後半に開発された,革新的な組版システムである.



この{\TeX}を用いれば書式が統一されてより美しい文書を作成する事
ができます.\Z{オフィスソフト}よりも論理構造のきちんとした文章が
作成できます.無論{\TeX}がそれらの処理を自動でやってくれるわけではな
いので,ユーザが必要な命令を明示的に記述します.覚えるまでに多少時間
がかかるのが難点ですが,慣れれば文書の編集には便利なツールです.

\section{WYSIWYGとは何か}\index{WYSIWYG}
\ruby{WYSIWYG}{ウイジイウイグ}とは
\qq{What You See Is What You Get}の略で
\yo{見たままのものが得られる}という意味
合いでワープロソフトのように画面で見たイ
メージがそのまま紙などに出力される事を言います.
{\TeX}はWYSIWYGではありませんから紙に出力されるイメージをどうにかして
確認する作業が必要になります.毎回紙に印刷するのは大変時間を必要とし,
なおかつ地球環境の悪化を促進するものです.そのためコンピュータ
の画面上で確認作業をします.これを\Z{プレビュー}と言います.
%なんだか面倒な手順を踏んでいるように思えますがこれはWYSIWYGに比べると
%応用が広いのです.
%WYSIWYGはときに\qq{What You See Is all You've Get}と言われることもあり,
%\yo{見たままのものしか得られない.}という意味合いで使われます.{\TeX}とは
%まったく正反対のシステムのことを指します.

\section{一括処理とは何か}
\zindind{原稿}{の処理}%
{\TeX}のもう一つの特徴として通常の\Z{プログラミング言語}と同じように
原稿を一括で処理する方式を採用しています.これは当然の事なのですが
ワープロソフトとは大違いです.\Z{一括処理}(\Z{バッチ処理})を
採用しているという事で,仕上がりは\K{全てのページの組版が終了
するまで}分からないという事です.マークアップ方式の言語ならば文書の
全体をフォーマット\pp{マークアップ付け}しなければならないのです.
\indindz{文書}{マークアップ言語による}%
%{\TeX}のもう一つの特徴として通常の\Z{プログラミング言語}と同じように
%原稿を一括で処理する方式を採用しています.これは当然のことなのですが
%ワープロソフトとは大違いです.一括処理を採用しているということは,
%仕上がりは\emph{全てのページの組み版が終了するまで}分からないと
%いうことです.マークアップ方式の言語ならば文書の全体を
%フォーマット\pp{マークアップ付け}しなければならないのです.

\section{\texorpdfstring{\LaTeX}{LaTeX}とは何か}

組版プログラムとしての{\TeX}は完成度が非常に高く,高性能です.
そのためちょっとした記事を書こうと思っても手続きが非常に
多いようです.そこであらかじめいくつかの命令を定義しておき,
その定義を使って特定の書式を用意しておけば簡単に文書を
作成する事ができます.このシステムを開発されたのが%
\Person{Leslie}{Lamport}で,彼の作成したシステムを\Prog[LaTeX]{\LaTeX}と言います.
%これも発音が微妙で,読み方には何種類かあります.
%作者のLamportが言う分には\yo{らてふ}と発音するのが
%正しいと思いますが,日本では広く\yo{らてっく}と発音されていま
%すので\yo{らてっく}でも良いと思います.

%{\LaTeX}もHTMLと同様のマークアップ方式を採用しています.
%簡単な例を挙げると
%\begin{InText}
% <CENTER>
%   人類普遍の原理である
%</CENTER>
%\end{InText}
%などがあります.これは\yo{人類普遍の原理である}
%という文字列を中央に寄せたいので,
%始まりと終わりをそれぞれ,
%\qu{\str{<CENTER>}}と\qu{\str{</CENTER>}}という
%二つの規則で文字を囲んでいます.
%これがマークアップ方式の典型的な例です.
%マークアップ方式ではそれぞれの要素に属性を与えて
%文書を記述するということを行います.
%これを{\LaTeX}で示せば
%\begin{InTeX}
%\begin{center}
%  人類普遍の原理である
%\end{center}
%\end{InTeX}
%となるので,先程のHTMLの記述に良く似ているのが
%お分かりになるでしょう.
{\LaTeX}も\Z{HTML}と同様のマークアップ方式を採用しています.
簡単な例を挙げると,次のような記述があるとします.

\begin{InText}
<CENTER>
  人類普遍の原理である
</CENTER>
\end{InText}

これは\yo{人類普遍の原理である}と
いう文字列を中央に寄せたいので,「始まり」と「終わり」をそれぞれ,
\qu{\str{<CENTER>}}と\qu{\str{</CENTER>}}という二つの規則で囲ん
でいます.これがマークアップ方式の典型的な例です.マークアップ
方式ではそれぞれの要素に属性を与えて
文書を記述するという事を行います.

\begin{InText}
\begin{center}
   人類普遍の原理である
\end{center}
\end{InText}

HTMLでの表記が{\LaTeX}ではこのようになるので,
先程のHTMLの記述に良く似ているのが,お分かりになるでしょう.


{\TeX}も{\LaTeX}も欧文言語圏のためのプログラムですから
標準では日本語を処理する事ができませんが,\Hito{中野}{賢}を
始めとするアスキーの方々が{\TeX}の日本語化をしてください
ましたので,今ではこの{\laTEX}を使って高品質な日本語組版が
できるようになりました.アスキーによって日本語化された%
\index{pTeX@\pTeX}%
\index{pLaTeX@\pLaTeX}%
{\TeX}や{\LaTeX}をそれぞれ{\pTeX},{\pLaTeX}と呼びます.

\index{LaTeX2.09@\LaTeX\,2.09}%
\index{LaTeX2e@\LaTeXe}%
\index{LaTeX 3@\LaTeX\,3}%
{\LaTeX}が最初に登場したときのバージョンがあり,
この頃のものを{\LaTeX\,2.09}と区別しています.それから煩雑で
あった{\LaTeX\,2.09}システムを整備して{\LaTeXe}が
{\LaTeX}プロジェクトチームによってリリースされました.
次の{\LaTeX}のバージョンは{\LaTeX\,3}と呼ばれていますが,
このバージョンが登場するのはもう少し先のようです.


\section{\LaTeX の導入}

\LaTeX の導入に関しては可能であれば近くにいる詳しい方にインストール方法
を聞いて導入した方が無難です.もし個人的に導入するのであれば,環境によっ
て次のようにインストールする事になります.なるべくウェブから最新版の
\TeX 環境を導入するようにし,可能であれば定期的に更新してください
\footnote{現在 \TeX のシステムを更新するための指針が提示されてい
ますが,本書では詳しく扱いません.簡単に説明すると複数の\str{texmf}ツリー
と呼ばれるディレクトリを用意し,配布されている\TeX のファイルと自分が
後から追加したファイルを分離する,というような事が可能になる管理方法が
あります.}.
\begin{description}
 \item[Windows] 
 \zindind{Windows}{への導入}%
 \Hito{阿部}{紀行}\footnote\webAbenori による『\TeX インストーラ3』を用いると非常に
 簡単に \TeX に関わるソフトウェア(角藤版\TeX, \Dviout, \GS, GSView,
 \Y{jsclasses})を導入する事ができます.このインストーラについては,例えば
 \Hito{大友}{康寛}による『ワープロユーザーのための\LaTeX 入門』にある
 インストールの解説\footnote{\webOtomoTeX}を参照してみてください.

 \item[Mac OS X]  
\zindind{Mac OS X}{への導入}%
 \Prog{MacOS X WorkShop}\footnote\webTaizo
 %\Prog{EasyPackage}\footnote{\webeasypackage} 等
 で簡単に周辺ツールも導入できます\footnote{\Z{X11}も導入していれば,
 GUIインタフェースの\Prog{Synaptic}によるパッケージ管理も可能となります.}.
 今後の展開については \Z{MacWiki}\footnote{\webMacwiki}等を参照してくだ
 さい.

 \item[Vine Linux] 
\zindind{Vine Linux}{への導入}%
 コンソールから管理者権限で \type{apt-get install task-tetex}
 と実行するだけで\TeX 関係のパッケージが導入されます.
% これで終わりかいな,びっくりちょんまげ。
 \item[Fedora Core]
\zindind{Fedora Core}{への導入}
 \Hito{土村}{展之}による ptetex3 において Fedora~Core~5 用の RPM が
提供されています\footnote{\url{http://tutimura.ath.cx/~nob/tex/ptetex/ptetex3/rpm/}}.
\end{description}

\LaTeX の導入と周辺情報に関しては\hito{奥村}{晴彦}による
\TeX~Wiki\footnote{\webTeXWiki}%
を参照するのが良いと思います.数ヶ月古いというだけで何らかの問題が発生す
る可能性もありますから,可能な限りインターネットから最新の\LaTeX を導入
するようにしてください\footnote{書籍の付録として\TeX 環境を提供すると,
逆にユーザにトラブルの原因を増やしてしまいかねないため,本書にはそのよう
な類いのものは付与しません.}.


Emacsのようなテキストエディッタやコンソールからの操作等に慣れていない方
は,\TeX 環境とは別に,\TeX の執筆支援環境も導入すると大変便利かと思われ
ます.\secref{basic:lakulaku}を参照し,それぞれの環境に応じて適切だと
思うプログラムを導入してみてください.


%\section{情報の入手先}
%一般に{Lamport}の\wasyo{\LMANUAL}~\cite{latexbook}や
%コンパニオンシリーズ~\cite{latexcomp,graphicscomp,webcomp},
%入門用として\Hito{奥村}{晴彦}の\wasyo{{\LaTeXe}美文書作成入門}%
%~\cite{bibunsyo3},それに\Hito{藤田}{眞作}の%
%書籍~\cite{latex2ecommand}や,\Hito{乙部}{厳己}と\Hito{江口}{庄英}
%による\yousyo{Another Manual}シリーズ~\cite{anothermanual1,anothermanual2,%
%anothermanual3}が参考になると思います.以上の書籍は入手が容易だと
%思います.
%
%\appref{info}に詳細な情報の入手先をまとめましたので,
%そちらからウェブのリンクなどを辿ってみてください.

\begin{comment}

\section{より良く学ぶために}
とにもかくにも\Person{Leslie}{Lamport}が書いた\yousyo{{\LaTeX} A Document 
Preparation System}~\cite{latexbook}という本を読みましょう.
この本を読めば{\LaTeX}の基本はほとんど習得する事ができます.
\yo{うーん,でも{\LaTeX}ってフリーウェアなんでしょ?フリーマニュ
アルじゃないじゃん,あががが.}とどこからか聞こえてきました.
まあしょうがないじゃないですか,そこは勘弁してくださいよ.
きちんとした本として出すためには出版社と独占契約をしないと
いけないんですから.愚痴を言っても始まらないので次に進みましょう.

{\LaTeX}の拡張的なことは150以上のマクロパッケージを
解説した\Person{Michel}{Goossens}らが書いた
\wasyo{{\LaTeX} コンパニオン}~\cite{latexcomp}という本を
読みましょう.これは原書で第2版があります.
うーん,確かにマクロパッケージを紹介する本としては
できが良いのですが\yo{これも買わないとあかんのですか?ぴよぴよぉ.}
とまた空から声が聞こえてきます.まぁ,買ってあげてください.
この本の収入が{\LaTeX}プロジェクトへの貢献にもつながるのですから.

さらに画像操作に関しては\Person{Michel}{Goossens}らが書いた\wasyo{{\LaTeX}
グラフィックス コンパニオン}~\cite{graphicscomp}という本を読みましょう.
ほほぅ,これは{\PS}や{\LaTeX}における画像操作について
詳しく書いてある本なのですが,またまた\yo{わしはそんなに本
を買ってられないぞ!}と怒りにも似た声が聞こえてきます.

極めつけに\Z{World Wide Web}で{\LaTeX}を扱うためには
\Person{Michel}{Goossens}らが書いた\wasyo{{\LaTeX} Web コンパニオ
ン}~\cite{webcomp}という本を読みましょう.
本当に詳しく解説しているのですが\yo{わしはもう疲れたわぁ.}
というとどめを刺されたような声が聞こえます.

ふむふむ,疲労困ぱいしているようですが
まだまだこれからですよ,頑張ってついてきてください.

\end{comment} 
