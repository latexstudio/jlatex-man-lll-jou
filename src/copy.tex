\begin{comment}
\section{著作権について}
参考文献の出力という前に著作権について考えてみましょう.
本書の冒頭には\Person{Richard}{Stallman}が書いた
\wasyo{フリーウェアとフリーマニュアル}という文書が転載
されています.この文書には現代のフリーウェアの抱える問題点
と著作権について言及しているので読んでみると良いでしょう.

著作権法第~1~章第~1~節の通則\pp{目的}の第1条には
\begin{quote}
この法律は,著作物並びに実演,レコード,放送及び有線放送
に関し著作者の権利及びこれに隣接する権利を定め,これらの
文化的所産の公正な利用に留意しつつ,著作者等の権利の保護
を図り,もつて文化の発展に寄与することを目的とする. 
\end{quote} 
と書かれています.\yo{レコード}いうのがなんとも古臭いの
ですがそれは置いといて,著作物とはどんなものを言うのでしょうか.
第~2~条\pp{定義}では
\begin{quote}
この法律において,次の各号に掲げる用語の意義は,当該各号に定
めるところによる. 
\begin{description}
\item[一 著作物] 
  思想又は感情を創作的に表現したものであつて,
  文芸,学術,美術又は音楽の範囲に属するものをいう. 
\item[二 著作者] 
  著作物を創作する者をいう.
\item[三 実演]
  著作物を,演劇的に演じ,舞い,演奏し,歌い,口演し,
  朗詠し,又はその他の方法により演ずること\pp{これらに
  類する行為で,著作物を演じないが芸能的な性質を有す
  るものを含む.}をいう. 
\item[四 実演家] 
  俳優,舞踊家,演奏家,歌手その他実演を行なう者及び
  実演を指揮し,又は演出する者をいう. 
\item[五 レコード] 
  蓄音機用音盤,録音テープその他の物に音を固定したもの
\pp{音をもつぱら影像とともに再生することを目的とするもの
  を除く.}をいう. 
\item[六 レコード製作者]
  レコードに固定されている音を最初に固定した者をいう. 
\item[七 商業用レコード] 
  市販の目的をもつて製作されるレコードの複製物をいう. 
\end{description}
以下省略.
\end{quote}
などの定義があるわけです.\K{学術的な創作物}に対しては
著作権というものが存在して著作者は保護されるように
なっているようです.さらに主な著作物として
第~2~章第~1~節第~10~条に
\begin{quote}
 この法律にいう著作物を例示すると,おおむね次の通りである. 
\begin{description}
 \item[一]小説,脚本,論文,講演その他の言語の著作物 
 \item[二]音楽の著作物 
 \item[三]舞踊又は無言劇の著作物 
 \item[四]絵画,版画,彫刻その他の美術の著作物 
 \item[五]建築の著作物 
 \item[六]地図又は学術的な性質を有する図面,図表,模型その他の図形の著作物 
 \item[七]映画の著作物 
 \item[八]写真の著作物 
 \item[九] プログラムの著作物 
\end{description}
\end{quote}
という明記があります.但し,
\begin{quote}
事実の伝達にすぎない雑報及び時事の報道は,
前項第一号に掲げる著作物に該当しない.
\end{quote}
とあるので,新聞などの事実報道には著作権はないのです.
ただし,\K{編集著作物}という項目があり第12条で
\begin{quote}
編集物\pp{データベースに該当するものを除く.以下同じ.}で
その素材の選択又は配列によつて創作性を有するものは,
著作物として保護する. 
\end{quote}
とありますから,レイアウトされたものには著作権があるようです.

第2章第3節では著作権で著作者にどのような権利があるかが
書かれているのですが,大まかに
\begin{description}
 \item[公表権] 世の中に自分の著作物を公表する権利.
 \item[氏名表示権] その著作物がどこの誰が作ったのかを提示する権利.
 \item[同一性保持権] 勝手に自分の著作物を改変されたり変更されない権利.
\end{description}
の三つが権利として成立します.以上のことを踏まえると
自分の論文などに他人の論文や書籍を引用する場合はそれらの権利を
侵害してはいけないようにするということです.しかし,
保護ばかりしていては学術的発展は望めませんので
第~2~章第~3~節第~5~款において
\begin{description}
\item[第~30~条 私的使用のための複製] 
著作物を保持する個人が個人的かまたは家庭内でその
著作物を複製する権利がある.
\item[第~31~条 図書館などにおける複製] 
図書館などの利用者は営利を目的とせずに調査研究のためならば
著作物の1部分を複製できる.
\item[第~32~条 引用] 第~32~条の該当箇所を引用すると
\begin{quote}
公表された著作物は,引用して利用することができる.
この場合において,その引用は,公正な慣行に合致する
ものであり,かつ,報道,批評,研究その他の引用の
目的上正当な範囲内で行なわれるものでなければなら
ない. 
\end{quote}
と記されています.学術目的の引用は著作憲法で認められて
いるわけです.
\end{description}
さらに教育機関の関係者は第~35~条\pp{学校その他の教育機関における
複製}という項目も当てはまります.
\begin{quote}
学校その他の教育機関\pp{営利を目的として設置されているものを除く.}
において教育を担任する者は,その授業の過程における使用に供すること
を目的とする場合には,必要と認められる限度において,公表された著作
物を複製することができる.ただし,当該著作物の種類及び用途並びにそ
の複製の部数及び態様に照らし著作権者の利益を不当に害することとなる
場合は,この限りでない.
\end{quote}
というのが第~35~条なのですが,\K{その授業の過程における使用
に供することを目的とする場合}だけにしか適用されないという
ことです.ですから生徒が友達から借りた教科書をコピーするのは
私的複製の範囲を超えているわけです.

さて,もっとも重要な部分は著作権法の第~48~条\pp{出所の明示}に
あります.これは
\begin{quote}
次の各号に掲げる場合には,当該各号に規定する著作物の出所を,
その複製又は利用の態様に応じ合理的と認められる方法及び程度
により,明示しなければならない. 
\begin{description}
\item[一]
\K{第三十二条},第三十三条第一項(同条第四項において準用する
場合を含む.),第三十七条第一項若しくは第三項,第四十二
条又は第四十七条の規定により著作物を複製する場合 
\item[二]
第三十四条第一項,第三十七条の二,第三十九条第一項又は第
四十条第一項若しくは第二項の規定により著作物を利用する場合 
\item[三]
第三十二条の規定により著作物を複製以外の方法により利用する
場合又は第三十五条,第三十六条第一項,第三十八条第一項,第
四十一条若しくは第四十六条の規定により著作物を利用する場合
において,その出所を明示する慣行があるとき. 
\end{description} 
\end{quote}
という内容になっており,\K{引用する場合はその出展を明記する}
ということがこの法律によって決められています.ですから
誰かの文章を引用した場合はその出典を明記することは
絶対に必要なことなのです.参考文献を読者に告げるのは
当然のことであって,引用の形にも配慮しなければならないわけです.
そうしないと文化の発展も学術的な発展も思うように行かないわけです.

ただし,著作権にも期限があり第~2~章第~4~節第~51~条 
\pp{保護期間の原則}では
\begin{quote}
著作権の存続期間は,著作物の創作の時に始まる. 
\begin{description}
 \item[2] 著作権は,この節に別段の定めがある場合
を除き,著作者の死後\pp{共同著作物にあつては,最
終に死亡した著作者の死後.次条第一項において同じ.}
五十年を経過するまでの間,存続する. 
\end{description} 
\end{quote}
なる項目がありますから,50年を経過した著作物には
その権利がないわけです.夏目漱石や福澤諭吉の作品を
改変したりすることは自由なわけです.ただし,その2次的
著作物の出典を明記するのはマナーとして行ったほうが無
難です.

以上の著作権法に関しては\yo{社団法人 著作権情報センター}
のウェブページ
\begin{quote}
\url{http://www.cric.or.jp/}
\end{quote}
から情報を得ました.著作権法についてもっと知りたいと
思ったならば参照してみてください.
\end{comment}
