%#!platex jou.tex
\chapter[フリーソフトウェアとフリーマニュアル]
   {{\huge フリーソフトウェアとフリーマニュアル}}\chaplab{free}
\markboth{フリーソフトウェアとフリーマニュアル}
{フリーソフトウェアとフリーマニュアル}

\begingroup
\small
%\setlength{\evensidemargin}{\oddsidemargin}
%\setlength{\textwidth}{\fullwidth}
%\setlength{\linewidth}{\fullwidth}
\begin{multicols}{2}

フリーなオペレーティングシステムにおける最大の欠陥は、ソフトウェアの問題
ではありません\zdash 私たちがこれらのシステムに含めることが可能な、フリーの良
質なマニュアルが不足していることこそが問題なのです。私たちの最も重要なプ
ログラムの多くは、完全なマニュアルと共に提供されていません。ドキュメント
はいかなるソフトウェアパッケージにおいても必要不可欠な一部分ですから、重
要なフリーソフトウェアパッケージがフリーなマニュアルと共に提供されないな
らば、それは大きな欠陥です。私たちは今日、そのような欠陥を数多く抱えてい
ます。 

昔々、何年も前のことになりますが、Perl を学ぼうと思ったことがあります。
私はフリーなマニュアルを一部入手したのですが、それは極めて読みにくいもの
でした。Perl ユーザたちに代替品について聞いてみたところ、彼らはより良い
入門用マニュアルがあると教えてくれたのですが、しかしそれらはフリーではあ
りませんでした。 

これはどうしてだったのでしょう? その良質なマニュアルの著者たちは、マニュ
アルをを O'Reilly Associates 社のために書き、O'Reilly はそれらを制限的な
条件の下で出版したのです。禁複製で禁変更、ソースファイルは入手不可\zdash
こういった条件はマニュアルをフリーソフトウェアのコミュニティから締め出して
しまいます。 

これはこの種の出来事としては最初のものではありませんでした。そして(私た
ちのコミュニティにとっては大きな損失なのですが)最後であるとも到底いえま
せん。この出来事以降も、独占的なマニュアル出版者は非常に数多くの著者たち
をそそのかし、彼らのマニュアルに制限を加えさせてきました。私は、 GNU ユー
ザの一人が彼の書いているマニュアルについて熱心に語るのを何度も聞きました。
彼はそれによって GNU プロジェクトを援助できると考えていたのです\zdash ところ
が、彼は続けて、私たちがそれを使うことができないように制限を課すであろう
出版者との契約にサインしたと述べたので、私の希望は打ち砕かれるのが常でし
た。 

きちんとした英語で書くということはプログラマの間ではまれなスキルですから、
マニュアルをこういったことで失う余裕は全く無いのです。 

フリーな文書で問題となるのは、フリーソフトウェアと同様、自由であり、価格
ではありません。これらのマニュアルの問題点は O'Reilly Associates が印刷
されたコピーに代価を要求するということではないのです\zdash それ自体は別に構い
ません(フリーソフトウェア財団も、フリーなGNU マニュアルの印刷されたコピー
を販売しています)。しかし、GNU マニュアルはソースコード形式で入手可能な
のに対し、O'Reilly のマニュアルは紙媒体でしか入手できません。GNU マニュ
アルは複写および変更の許可と共に提供されています。Perl のマニュアルはそ
うではありません。こういった制限こそが問題です。 

フリーなマニュアルであるための基準はフリーソフトウェアのそれとかなり良く
似ています。その基準とは、すべてのユーザにある種の自由を与えるということ
です。マニュアルをオンラインまたは紙媒体でそのプログラムのすべてのコピー
と一緒に提供できるよう、再配布(商業的再配布を含む)が許可されていなければ
なりませんし、変更の許可も重要です。 

普遍的なルールとして、人々にあらゆる種類の文章や書籍を変更する許可を与え
ることが必須だとは私も思いません。書かれたものに関する論点が、ソフトウェ
アのそれと同じである必要は無いのです。例えば、あなたや私に、この文章のよ
うな、自分の行動やものの考え方を説明する論説を変更する許可を与える責任が
あるとは考えられません。 

しかし、フリーソフトウェアのための文書にとって変更の自由がなぜ重要である
かについては、特別の理由があるのです。人々がソフトウェアを変更する権利を
行使して、ソフトウェアに機能を加えたり変更したりすると、彼らが良心的であ
ればマニュアルも変更しようとするでしょう\zdash そうすれば彼らは正確で有用なド
キュメントを変更されたプログラムと共に提供できるわけです。そこで、プログ
ラマが良心的になることや仕事を完遂することを禁止する、あるいはより正確に
言えば、プログラムを変更するなら彼らがゼロから新しいマニュアルを書きなお
すことを要求するマニュアルは、私たちのコミュニティのニーズを満たすものと
は言えないのです。 

全面的な変更の禁止は受け入れられませんが、ある種の制限を変更の手法に課し
ても何の問題も引き起こしません。例えば、原著者の著作権表示を保存すること
を要求したり、元の配布条件や著作者名のリストの保存を要求するのはOK でしょ
う。また、変更された版は、それらが変更されたという告知を含んでいなければ
ならないという要求をするのも構いませんし、ある節全体の削除や変更を禁止す
ることすらも、そういった節が技術的なトピックを扱っていない限り問題にはな
りません(GNU マニュアルの一部はそういった節を含んでいます)。 

この種の制限が問題にならないのは、現実的な問題としては、それらが良心的な
プログラマがマニュアルを変更されたプログラムにあわせて手直しすることを止
めさせないからです。言い換えれば、そういった制限はフリーソフトウェアのコ
ミュニティがマニュアルを最大限に活用することを禁止しないのです。 

しかしながら、マニュアルの技術的な内容はすべて変更可能でなければなりませ
ん。そして、変更の結果をあらゆる一般的なメディア、すべての通常チャンネル
で配布できなければなりません。そうでなければ制限はコミュニティを妨げます
ので、マニュアルはフリーではなく、そこで私たちには他のマニュアルが必要と
なります。 

不幸にも、独占的なマニュアルが存在する場合には、もう一つマニュアルを書い
てくれる人を探すのは困難なことが多いのです。障害となるのは、多くのユーザ
が独占的なマニュアルで十分と考えていることです\zdash そこで、彼らはフリーなマ
ニュアルを書く必要を認めないのです。彼らはフリーなオペレーティングシステ
ムが、満たすべき欠落を抱えていることが分からないのです。 

どうしてユーザは独占的なマニュアルで十分だと思うのでしょう? 何人かは、こ
の問題について考えたことがないのでしょう。私はこの論説が、そうした状況を
いくらかでも変えることを期待しています。 

他のユーザは、独占的なマニュアルは独占的なソフトウェアが受け入れられるの
と同じ理由から受け入れられるものだと考えます。彼らは全く実用的な見地から
のみ物事を判断し、自由を基準として適用しないのです。こういった人々にも彼
らなりの意見を持つ資格がありますが、しかしそういった意見は自由を含まない
価値から飛び出してくるものなので、彼らは自由を評価する私たちの基準のより
どころとはなり得ません。 

この問題についての話を広めてください。私たちはマニュアルを独占的な出版の
ために失い続けています。もし私たちが独占的なマニュアルは十分では無いとい
う思想を広めるならば、おそらく GNU を文書を書くことで助けたいと思う次の
人は手遅れになる前に、彼らがそれを結局のところフリーにしなければならない
ということを悟るでしょう。 

私たちはまた、商業的出版者に対して、独占的なマニュアルに代わってフリーで
コピーレフトの主張されるマニュアルを販売することを薦めています。あなたが
この動きを援助できる一つの方法は、マニュアルを買う前にその配布条件をチェッ
クし、コピーレフトなマニュアルを非コピーレフトなものよりも好んで買うとい
うことです。 

\begin{center}
 Copyright \textcopyright\ 2000 Free Software Foundation, Inc., 59
 Temple Place - Suite 330, Boston, MA 02111, USA
\end{center}


本文に一切変更を加えず、この著作権表示を残す限り、 この文章全体のいかな
る媒体における複製および配布も許可する。

\end{multicols}

\endgroup

%\clearpage

\section*{Free Software Foundationとその活動ついて}

前述の『フリーソフトウェアとフリーマニュアル』は\emph{Free Software
Foundation}の\Person{Richard}{Stallman}によって書かれたウェブページを
\Hito{八田}{真行}\footnote{mhatta@gnu.org}が翻訳したものです.この
\ruby{GNU}{ヌー}に関する文章を公開している団体をFSF: \emph{Free Software
Foundation}\footnote{gnu@gnu.org}と言います.彼らが目指す社会,彼らの思
想の詳しい事についてはウェブページ\footnote{\webGNU}を
アクセスすると良いでしょう.私が本書を作ったきっかけも,このFSFの活動に
触発されたものですから,興味がありましたらご覧ください.

% title: \emph{The Design Philosophy of Hermann Zapf}
% author: \Person{Hermann}{Zapf}
% ISBN:  4-947613-15-7
% year: 1995 {\textcoyright} Robundo
% 日本にも書道という物がある.書家の精神性が書に表現され,その芸術性が高
% く評価されている.西洋では美術的価値が認知されていない場合がある.

\index{GNU!\zdash FDL}
もちろん,この本も GNU FDL\footnote{FSF によるフリーな文書の利用に関する
ライセンスの事です.}で発行されています,印税免除で\ldots.

\begin{comment}

\section*{ずるいビジネスマンに利益を横取りされる世の中}

%\begin{metacomment}
% 現代社会で出版という道を辿るためには,出版社との不利な契約を強いられる
% 場合が多い.実際,独占契約で利益配当は出版社に一方的に決められ,締め切
% りもページ数も何もかも出版社が先導的に押し付けるケースも存在するだろう.
%\end{metacomment}

ずるいビジネスマンに自分の著作物の利益を横取りされる世の中になっている
のはどうしてか.それは何か高尚な理由,例えば公共奉仕とか社会的活動とい
うものによらず,専ら利益がビジネスマンを動かしているからに他ならない.
結局は利益が欲しいがために著作物を出版させているのである.社会のために
出版業務を積極的に行っているのは少数である.出版というのはある程度のリ
スクを伴う.本が売れなければ収入は減る.これは現実である.売れる本を出
版しなければ話にならない.

昔の著作権保護の方法というのはきわめて簡単なものだった.印刷所,書体,
紙,人員などあらゆるものが物理的保護に結びついていた.現代社会ではそのよ
うな物理的保護は必ずしも有効ではなく,電子媒体であれば数秒で片が付くケー
スもある.これによって社会はどのような方向に向かうべきだろう.電子媒体
には電子署名や電子透かし,コピープロテクトを施せば良いのだろうか.
私はこのような保護をしてもほとんどの場合はイタチごっこに終わるだろうと
考えている.事実,CD-ROM/DVD-ROM等のコピープロテクトも技術的知識のあ
る人が,その保護を回避するような手法を考えだしているし,保護を無効にす
るような装置も充実している.

社会全体が創造的な文化活動を望むのであれば,これを国家ではなく,民間の
非営利団体が管理すべきである.これによって利益の分配の平等性なども確保
し,若いクリエイターの活動も制限しなくて済むのである (cf.~J*SR*C).

しかし,これを実現するには少し高尚な集団が結成されている必要がある.利
益に駆り立てられた人間ではなく,本当に技術,科学,文化の発展のために献
身出来るだけの理念をもった人間である事が条件となる.

いつまで経っても私たちの社会は国家が音楽活動や出版活動,学術活動等の大
きい意味での文化活動を管理している限り,豊かな文化にはならないのである.

「カラオケは日本の文化です」というピー(放送禁止用語)な発言からも明白
ですが,要するに「あぁ,自分で創造・発展できない世の中なのね,日本は」
ということ.

\end{comment}
