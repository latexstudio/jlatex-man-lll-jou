%#!platex jou.tex
\chapter{変更履歴}\applab{changelog}

% 変更履歴専用の list 型環境 changelog
\newenvironment{changelog}
 {\list{}{%
     \setlength{\leftmargin}{3zw}%
     \setlength{\labelwidth}{0pt}%
     \setlength{\labelsep}{0pt}%
     \setlength{\itemsep}{0pt}%
     \setlength{\parsep}{0pt}%
     \let \makelabel = \releaselabel}}
  {\endlist\ignorespacesafterend}

\makeatletter
\newcommand*{\releaselabel}[1]{%
  \if @#1@%
    \global\@itempenalty\z@
  \else
    \global\@itempenalty\@M
    \hspace{-\leftmargin}\sffamily
    \@releaselabel #1@%
  \fi}
\def\@releaselabel#1 #2@{%
  \makebox[\leftmargin-0.5em][r]{#1}%
  \hspace{0.5em}#2}
\makeatother

この文書は私一人で執筆しておりますから,どこかに間違いや誤植がある確率が
高くなっています.「あれっおかしいな?」と思う箇所がありましたら私のホー
ムページ\footnote{\webThorTeX}の掲示板かメールアドレス
\footnote{thor@tex.dante.jp}にご連絡ください.

\begin{comment}
変更履歴の執筆にあたっては,1--5 を区別して記述すること.
 \begin{itemize} 
  \item[削除 (deleted)]   ある要素を取り除く事.
  \item[追加 (added)]     ある要素を新規に取り入れる事.
  \item[加筆 (extended)]  すでにある要素の小要素を追加する事.
  \item[修正 (revised)]   すでにある要素に変更を加える事.
  \item[改訂 (updated)]   文書を改めて訂する事.
  \item[誤植 (fixed bug)] 誤った記述.訂正されるべき要素.
 \end{itemize}
\end{comment}

\begin{changelog}
 \item[1.14 2016/07/23] % 1.14
   \item 大石勝様のURLを訂正.

 \item[1.13 2006/08/18] % 1.13
   \item 誤植訂正版.

 \item[1.12 2006/05/12] % 1.12
   \item \figref{RingServer}における敬称が省略されていたのを修正しました.
   \item 倍角ダーシの後の余計な空白を取り除きました.
   \item \secref{ref:macropackage}の\sty{fancyhdr}パッケージのアドレスを
   修正しました.
 \item[1.11 2006/05/12] % 1.11
   \item ver.~1.10 をさらに校正しました.
   \item 前付けにある『フリーソフトウェアとフリーマニュアル』を削除しました.
   これに伴い『謝辞』の直後のページに代わりとなるFSFとこの文書の
   位置付け及びPDF版の本書の所在に関する説明を追加しました.
   \item 文書全体において語句・語調の統一を行いました.
 \item[1.10 2006/05/07] % 1.10
   \item ver.~1.00 の誤植訂正版として配布しました.
 \item[1.00 2006/04/20] % 1.00
   \item 大幅な改訂を行い,方向性を若干「理工系の学生・研究者向け」とし
   ました.
   \item ページレイアウトと使用しているマクロの改変を行いました.
   \item \secref{RingServer}を追加しました.
   \item \appref{info}の人名を包括的に索引に追加しました.
   \item 口絵を削除しました.
   \item 前付けにある『まえがき』の『凡例』を加筆し,『FUNNISTについて少
   し』を削除しました.
   \item 前付けにある『フリーソフトウェアとフリーマニュアル』の『Free
   Software Foundationとその活動ついて』を加筆しました.
   \item 章構成を入れ替えました.
   \item \secref{doublespace}を修正しました.
   \item \chapref{math}に例題を加筆しました.
   \item \secref{amsmath}に\AmSLaTeX に関する情報を追加しました.
   \item \secref{figure}を加筆・修正しました.
   \item \secref{lakulaku}を修正しました.
   \item \appref{future}を追加しました.
 \item[0.34 2005/03/20] % 旧 0.3d
   \item 0.33 の誤植訂正版という形で配布しました.
 \item[0.33 2004/12/28] % 旧 0.3c
   \item 誤植の訂正を行いました.
   \item ライセンス的に \emph{free} とは言いがたい画像を削除しました.
 \item[0.32 2004/01/14] % 旧 0.3b
   \item 誤植の訂正を行いました.
   \item 著者の連絡先が変更になったため,URL と e-mail を変更しました.
   \item URLの変更にともない\appref{info}を修正しました.
 \item[0.31 2004/08/19] % 旧 0.3a
   \item 誤植の訂正を行いました.
   \item 改行が変な部分や索引の倍角ダーシを修正しました.
   \item URLの変更等にともない\appref{info}を修正しました.
 \item[0.30 2004/08/05] % 旧 0.30
   \item 初級編に必要だと思われる部分を記述し,これ以上は修正しないとい
   う完成版に近いものを公開しました.
 \item[0.21 2004/04/30] % 旧 0.2a
   \item 誤植の訂正を行いました.
 \item[0.20 2004/04/16] % 旧 0.20
   \item 誤植の訂正を行いました.
   \item 句読点を全角のピリオド・コンマに統一しました.
   \item 爪掛けについては章見出しも出力するようにしました.
   \item \chapref{math}で空きに関する記述を加筆しました.
   \item 索引について抜けていた人名や語句の補充をしました.
 \item[0.10 2004/04/02] 
   \item 初版を発行しました.
\end{changelog}

