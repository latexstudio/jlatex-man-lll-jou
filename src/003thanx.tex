%#!platex jou.tex
\chapter{謝辞}

本書を作成するためには非常に多くの方々のご協力,ご助言がなければ実現が
難しかった事を容易に想像できます.

まず{\TeX}の作者である\Person{Donald}{Knuth}には最大の感謝を表さなければ
なりません.氏が\TeX という土台を作ってくれたお陰で,こんなにも素晴らし
い世界を体験する事ができた事に喜びを感じております.%『文芸的プログラミン
%グ』と呼ばれるように,プログラミングという行為そのものを芸術の域に高め,
%さらに

{\LaTeX}全般に関しては\Hito{秋田}{純一},\Hito{奥村}{晴彦},\Hito{野
村}{昌孝},\Hito{吉永}{徹美}より多くの事を学びました.出版,校正,デザ
インなどに関しては\Hito{木村}{健一}よりご助言をいただいたり,また書籍を
貸して頂きました.

{\LaTeX}の作者である\Person{Leslie}{Lamport},
{\LaTeXe}の開発をされた\Person{Frank}{Mittelbach},
\Person{Johannes}{Braams},\Person{David}{Carlisle},\Person{Michael}{Downes},
\Person{Alan}{Jeffrey},\Person{Sebastian}{Rahtz},\Person{Chris}{Rowley},
\Person{Rainer}{Sch\"opf},
{\TeX}の日本語化をして下さった\Hito{中野}{賢}とアスキーの方々,
Windowsに{\pTeX}を移植してくださった\Hito{角藤}{亮},
\Dviout を開発された\Hito{大島}{利雄}と\Hito{乙部}{厳己},
\prog{\BibTeX}の開発をされた\Person{Oren}{Patashnik},
\prog{MakeIndex}を開発・改良された\Person{Pehong}{Chen}と\Person{Nelson}{Beebe},
\Dvipdfm の作者である\Person{Mark}{Wicks},{\Dvipdfmx}の保守・管理をされておられる
\Hito{平田}{俊作}と\Hito{趙}{珍煥},{\PS}やPDFなどのページ記述言
語を作成されたAdobe社の方々,さらに,フリーウェア,マクロパッケージなど
の作成で,{\TeX}の分野において貢献された方々にも感謝いたします.

\hito{大友}{康寛}や\hito{田中}{健太}には本書の誤記を指摘していただき,さ
らに改善すべき箇所について議論していただきました.
\hito{永田}{善久}にはドイツ語表記について教えていただきました.

多くの方々が本書の作成に貢献して下さいました.本当にありがとうございます.
協力してくださった方々のためにも,本書が日本の \TeX コミュニティにおける
恒久的な財産として残り続ける事を切に望んでおります.

\clearpage
\thispagestyle{empty}
\null\vfill

\index{GNU!\zdash FDL}
\centerline{\headfont 
  この本は GNU FDLで発行されています! もちろん印税免除で\ldots.}

本書は \fdl\footnote{FSF によるフリーな文書の利用に関する
ライセンスの事です.}の書籍ですから,その原稿とPDF版を著者のウェブページ%
\footnote{\webThorTypo}で公開しています.
誤記・誤植や補足事項に関する情報を取り扱っています\footnote{書名は
『好き好き\LaTeXe 初級編』という名前で公開している場合があります.}.
%
本書と「同じような出力を \LaTeX で実現したい」と感じたので
あれば,直接原稿を参照してみてください.

本書の印刷用の PDF,\Y{hyperref}による便利な操作が可能な閲覧用 PDF も公
開しています.パソコンなどに閲覧用 PDF を保存しておけば文字列検索もでき
ますし,紙媒体の本書がない時にも活用できるものと思います.

最後に\gnu の思想\footnote{この\ruby{\gnu}{ヌー}に関する情報を公開してい
る団体をFSF: \emph{Free Software Foundation} (gnu@gnu.org) と言い
ます.彼らが目指す社会,彼らの思想の詳しい事についてはウェブページ
(\webGNU) にアクセスすると良いでしょう.私が本書を作ったきっかけも,この
FSFの活動に触発されたものです.興味がありましたらご覧ください.}と
\fdl を作成してくれた \emph{Free Software Foundation} の
\Person{Richard}{Stallman} に感謝します.



